\documentclass[a4paper]{article}

\usepackage[greek,english]{babel}
\usepackage{alphabeta}
\usepackage[dvipsnames]{xcolor}
\usepackage{listings}
\usepackage{tikz, tcolorbox}
\usepackage{geometry}
\usepackage{amsmath, mathtools}
\usepackage[bottom, symbol]{footmisc}

\renewcommand{\thefootnote}{\fnsymbol{footnote}}

\newcommand{\listsetup}
{\lstset{ 
  backgroundcolor=\color{blue!8!white},
  breaklines=true,
  commentstyle=\color{ForestGreen},
  frame=single,
  keywordstyle=\color{blue},
  language=C,
  numbers=left,
  numbersep=4pt,
  numberstyle=\color{black},
  rulecolor=\color{purple},    
  stringstyle=\color{orange},  
  showstringspaces=false,
  tabsize=4,
  framexleftmargin=14pt,
  framexrightmargin=0pt
}}

\newgeometry{top = 3cm,
			 bottom = 3cm, 
			 inner = 3cm,
			 outer = 2cm}
			 
\title{\textbf{Algorithms - Assignment 2}}

\author{Καζγκούτης Αθανάσιος \and
	    Παπαδόπουλος Δημήτριος-Λάζαρος
}



\begin{document}
\maketitle
\newpage
\section*{Πρόβλημα 1}

\listsetup
\begin{lstlisting}[mathescape]
function MaxSuccessPath(G,P,s)
    Initialize(G,s)
    D = $\emptyset$
    Q=G.V
    while Q$\neq\emptyset$
        u=ExtractMaxQ
        if (u = = t)
            u = t
            PrintPath(u)
        D = D $\cup$ {u}
        for each vertex v $\in$ G.Adj[u]
           Relax(u,v,p)
        
\end{lstlisting}

\listsetup
\begin{lstlisting}[mathescape]
function Initialize(G,s)
    for each vertex $\in$ G.V
        v.p = 0
        v.$\pi$ = NULL
    s.p = 1
\end{lstlisting}

\listsetup
\begin{lstlisting}[mathescape]
function Relax(u,v,p)
    if ($v.p < u.p \cdot p_{uv}$)
        $v.p = u.p \cdot p_{uv}$
        $v.\pi = u$
\end{lstlisting}
Η διαδικαίσα της ΧΑΛΑΡΩΣΗΣ μιας ακμης (u,v) έχει ως εξής:ελέγχουμε εαν μπορούμε να βελτιώσουμε την μέγιστη πιθανότητα (p)  της διαδρομής για τον κόμβο v διερχόμενοι μεσω του u .
H ποσότητα v.p είναι η πιθανότητα λήψης του μυνήματος χωρίς λάθη.
Η  ποσότητα v.π είναι ο προκάτοχως του v.
Ενώ το $p_{uv}$ είναι η πιθανότητα σωστης λήψης μυνήματος ανάμεσα στους κόμβους u,v

\listsetup
\begin{lstlisting}[mathescape]
function printPath(u)
    i = 1
    R[0] = u
    while $u.\pi \neq NULL$
        R[i] = $u.\pi $
        $u = u.\pi$
        i = i + 1 
    for j = i-1 : -1 : 0
        print $\rightarrow$ R[j]
\end{lstlisting}

Ο παραπάνω αλγόριθμος 



\newpage
\section*{Πρόβλημα 2}

\end{document}
