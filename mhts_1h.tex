\documentclass[a4paper]{article}

\usepackage[greek,english]{babel}
\usepackage{alphabeta}
\usepackage{listings}
\usepackage{tikz, tcolorbox}
\usepackage{xcolor}

\title{\textbf{Algorithms - Assignment 1}}

\author{Καζγκούτης Αθανάσιος \and
	    Charbel Al Haddad \and 
	    Παπαδόπουλος Δημήτριος-Λάζαρος
}

\date{March 24, 2023}

\begin{document}

\maketitle
\pagebreak
\section*{Πρόβλημα 1}

\begin{itemize}
\item Ερώτημα 1
\end{itemize}
O αλγόριθμος που παρατιίθεται παρακάτω δέχεται σαν είσοδο μια συστοιχία n στοιχίων όπου n ειναι οι ψήφοι -σε ονοματεπώνυμα-μιας κοινότητας.
Στόχος του αλγόριθμου ειναι:
\begin{enumerate}
    \item να ελέγξει ποιος/α έχει τους περισσότερους ψήφους.
    \item αμα ξεπερνούν ή είναι ίσοι του 50\%\ των συνολικών ψήφων.
    \item εφόσων ισχύει το παραπάνω επιστρέψει στην έξοδο το ονοματεπώνυμο του υποψήφιου(ελέγχοντας το ακραίο σεναρίο να υπάρχουν 2 υποψήφιοι που ισοβαθμούν ΚΑΙ έχουν το 50\%\ των ψήφων) .
  
\end{enumerate}
        $$ \textbf{ΑΛΓΟΡΙΘΜΟΣ 1}$$
\lstset{numbers=left, numbersep=5pt,frame=single,  framexleftmargin=15pt,}

\begin{lstlisting}[mathescape]
function MajorityFinder(A[1...n])
 majority_person = []
 maxcount = 0
 count
 temp
 for(i = 1 to n)
 	count = 0
 	temp = A[i]
        eq=1
 	for(j = 1 to n)
 		if(temp = = A[j])
 			count++
 	if(count > maxcount)
 		maxcount = count
 		majority_person[1] = temp
            if(eq>1)
            for(u=2 to eq)
 		majority_person[u] = null
            eq=1
            else break
 	else if(count = = maxcount){
             for(u=1 to eq)
             if (majority_person[u]=temp)
             break;
             else
             e++
 		majority_person[e] = temp
   }
 if(maxcount $\geq$ $\lceil\frac{n}{2}\rceil$)
 	return majority_person
 else 
 	return "no person has the majority"
\end{lstlisting}
\textit{ΑΝΑΛΥΣΗ ΑΛΓΓΟΡΙΘΜΟΥ:}

\begin{itemize}
    \item Στη 1η σείρα ο αλγόριθμος δέχεται τις n ψήφους μέσω μιας συστοιχίας "A[1...n]"
\end{itemize}
\begin{itemize}
    \item Στη 2η σειρα ορίζουμε τον πίνακα "majority person"στον οποίο θα αποθηκευτεί  το ονόμα του υπερέχoν υποψήφιου , το μέγεθος του πίνακα είναι μεταβλητό για να καλύπτει και την περίπτωση της ισοβαθμίας.
\end{itemize}
\begin{itemize}
    \item Στις γραμμές 3-5 γίνονται αρχικοποιήσεις που χρησιμέυουν στη καταμέτρηση των ψήφων και των υποψήφιων με τους περισσότερους.
\end{itemize}
\begin{itemize}
    \item Στις γραμμές 6-12 ο αλγόριθμος καταμετρεί όλες τις ψήφους .Συγκεκριμένα η αρχική η for χρησιμοποιείται για να προσπελαστούν όλες οι ψήφοι,ενώ η 2η για να γίνει έλεγχος ποιες ψηφοί έχουν το ίδιο ονοματεπώνυμο με την i-στη.Ετσι μεσω το count μετράμε τις συνολικες ψήφους που έχει ο υποψήφιος που βρισκεταιο στην i-στη θεση του αρχικου πινακα ενώ στο temp αποθηκέυται το ονοματεπώνυμο του.     
\end{itemize}
\begin{itemize}
    \item Στις γραμμες 13-27 γίνεται  ελεγχός για να βρέθει  ποιος υποψήφιος εχει τις περισσοτερες ψήφους,ενω καλυπτεται η περιπτωση της ισοβαθμιας υποψηφιων μεσω του μεταβλητου πινακα "majority person"
\end{itemize}
\begin{itemize}
    \item Στις τελευταιες γραμμες 29-32 γινεται ελεγχος αμα καποιος υποψήφιος εχει $\geq$ του 50\%\ των ψηφών,η εξοδος ειτε θα εχει κανενα ενα ή ακομα και 2 ομνοματα στην ακραια περίπτωση που έχουμε 2 υποψηφιους με τις μισε ψηφους αμφοτεροι.
\end{itemize}

\begin{itemize}
\item Ερώτημα 2
\end{itemize}

\lstset{frame=none}

\begin{tcolorbox}[colback=blue!20!white,colframe=purple!60!white,title=\textbf{Merge sort}]
\begin{lstlisting}[mathescape]
function mergesort(a[1...n])
 if(n > 1)
 	return merge(mergesort(a[1...$\lfloor\frac{n}{2}\rfloor$]),
                    mergesort(a[$\lfloor\frac{n}{2}\rfloor$ + 1 ...n]))
 else
 	return a
\end{lstlisting}

\begin{lstlisting}[mathescape]
function merge(x[1...k], y[1...l])
 if(k = 0)
 	return y[1...l]
 if(l = 0)
 	return x[1...l]
 if(x[1] $\geq$ y[1])
 	return x[1] $\circ$ merge(x[2...k], y[1...l])
 else
 	return y[1] $\circ$ merge(x[1...k], y[2...l])
\end{lstlisting}
\end{tcolorbox}

\lstset{numbers=left, numbersep=5pt,frame=single,  framexleftmargin=15pt}

\pagebreak

\begin{lstlisting}[mathescape]
function MajorityFinder2(A[1...n])
 majority_person = []
 mergesort(A)
 for(i = 1 to n)
 	if(A[i] = = A[$\lceil\frac{n}{2}\rceil$ - 1 + i])
 		if(majority_person[1] = = null)
 			majority_person[1] = A[i]
 		else
 			majority_person[2] = A[i]
 return majority_person
\end{lstlisting}

\begin{itemize}
\item Ερώτημα 3
\end{itemize}

\begin{lstlisting}[mathescape]
function MajorityFinder3(A[1...n])
 majority_person = []
 HashMap T
 for(i = 1 to n)
 	if(T.search(A[i]) = = true)
 		T[A[i]] = T[A[i]] + 1
 	else 
 		T.put([A[i], 1)
 	if(T[A[i]] $\geq \lceil\frac{n}{2}\rceil$)
 		if(majority_person[1] = = null)
 			majority_person[1] = A[i]
 		else
 			majority_person[2] = A[i]
 return majority_person
\end{lstlisting}

\section*{Πρόβλημα 2}

\begin{itemize}
\item Ερώτημα 1
\end{itemize}

\lstset{frame=none}
\begin{tcolorbox}[colback=blue!20!white,colframe=purple!60!white,title=\textbf{Algorithm 1}]
Έστω πίνακας T με στοιχεία n θετικούς ακεραίους με εύρος [0,...,k] (k ακέραιος)
\begin{lstlisting}[mathescape]
 for i = 0,...,k do
	 H[i] = 0
 end for
 for j = 1,...,n do
	 H[T[j]] = H[T[j]] + 1
 end for
 for i = 1,...,k do
	 H[i] = H[i] + H[i - 1]
 end for
 for j = n,...,1 do
	 S[H[T[j]]] = T[j]
	 H[T[j]] = H[T[j]] -1
 end for
\end{lstlisting}
\end{tcolorbox}



\begin{itemize}
\item Ερώτημα 2
\end{itemize}

\end{document}
